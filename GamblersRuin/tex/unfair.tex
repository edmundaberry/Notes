\section{Unfair Gambler's Ruin}
\label{sec:unfair}

Now let's examine the unfair ($p \neq 0.5$) Gambler's Ruin
problem described in Section \ref{sec:intro}.
Recall that:
\begin{itemize}
  \item The gambler begins with a fortune of $S_0 = i$ dollars.
  \item The gambler plays a series of games, and in each of these games he can either win a dollar with probability $p \neq 0.5$ or lose a dollar with probability $1-p \neq 0.5$.  
  \item The gambler's winnings in game $t$ are $X_t$.
  \item The gamber's fortune after $n$ games is $S_n = S_0 + \sum\limits_{t = 1}^{n}X_{t}$.
  \item The gambler will stop when he runs out of money ($S_N = 0$) or obtains $N$ dollars ($S_N = N$).
\end{itemize}

Unlike the fair Gambler's Ruin, $\left\{S_n\right\}$ is not a martingale.
We will have to use a different expression.
In this case, we define a new variable, $\lambda$, and a new function $f\left(S_n\right)$:
\begin{align}
  \lambda &= \frac{1-p}{p} \\
  f(S_n) &= \lambda^{S_n}
\end{align}

We can show that $\left\{f\left(S_n\right)\right\}$, however, {\textit is} a martingale:
\begin{align}
  \mathbb{E}[f(S_n)] &= \mathbb{E}[\lambda^{S_n}] \\ 
  \mathbb{E}[f(S_n)] &= p \cdot \lambda^{S_{n-1}+1} + (1-p) \cdot \lambda^{S_{n-1}-1} \\ 
  \mathbb{E}[f(S_n)] &= p \cdot \left(\frac{1-p}{p}\right)\cdot\lambda^{S_{n-1}} + (1-p) \cdot \left(\frac{p}{1-p}\right)\cdot \lambda^{S_{n-1}}  \\
  \mathbb{E}[f(S_n)] &= (1-p) \cdot\lambda^{S_{n-1}} + p \cdot \lambda^{S_{n-1}}  \\
  \mathbb{E}[f(S_n)] &= \lambda^{S_{n-1}} \\ 
  \mathbb{E}[f(S_n)] &= f(S_{n-1}) 
\end{align}
So $\left\{f\left(S_n\right)\right\}$ is a martingale.

Once again, we would like to evaluate the probability that the gambler accumulates $N$ dollars 
and does not go bankrupt ($p'$).  We can evaluate that probability as follows:
\begin{align}
  \mathbb{E}[f(S_n)] &= p' \cdot \lambda^N + (1-p') \cdot \lambda^0 \\
  \mathbb{E}[f(S_n)] &= p' \cdot \lambda^N + (1-p') \\
  \mathbb{E}[f(S_n)] &= p' \cdot \left(\lambda^N - 1\right) + 1
\end{align}
The fact that $\left\{f(S_n\right)\}$ is a martingale allows
us to evaluate this probability using the optional stopping theorem:
\begin{align}
  \mathbb{E}[f(S_n)] &= \mathbb{E}[f(S_0)] \\
  \mathbb{E}[f(S_n)] &= \lambda^i
\end{align}
Combining this information, we find:
\begin{align}
  \mathbb{E}[f(S_n)] &= p' \cdot \left(\lambda^N - 1\right) + 1 = \lambda^i \\
   p' &= \frac{1-\lambda^i}{1-\lambda^N}
\end{align}

We would also like to find the expected number of games ($\tau$) it will take the 
gambler to either go bankrupt ($S_\tau = 0$) or reach his goal ($S_\tau = N$).
To do this, we will use Wald's equality.
Note that Wald's equality works for any stopping condition on 
any random series, not just martingales:
\begin{align}
  \mathbb{E}[S_\tau] &= i + \mathbb{E}[X]\cdot\mathbb{E}[\tau] \\
  \mathbb{E}[\tau] &= \frac{1}{\mathbb{E}[X]}\left(\mathbb{E}[S_\tau] - i\right)
\end{align}
We evaluate $\mathbb{E}[X]$:
\begin{align}
  \mathbb{E}[X] &= p - (1-p) \\ 
  \mathbb{E}[X] &= p\cdot(1-\lambda) \\
  \mathbb{E}[X] &= \frac{1-\lambda}{1+\lambda}
\end{align}
We evaluate $\mathbb{E}[S_\tau]$:
\begin{align}
  \mathbb{E}[S_\tau] &= p'\cdot N \\
  \mathbb{E}[S_\tau] &= \left(\frac{1-\lambda^i}{1-\lambda^N }\right) \cdot N 
\end{align}
We can now evaluate $\mathbb{E}[\tau]$:
\begin{align}
  \mathbb{E}[\tau] &= \frac{1}{\mathbb{E}[X]}\left(\mathbb{E}[S_\tau] - i\right) \\ 
  \mathbb{E}[\tau] &= \left(\frac{1+\lambda}{1-\lambda}\right)\cdot \left[\left(\frac{1-\lambda^i}{1-\lambda^N}\right) \cdot N  - i\right]
\end{align}
