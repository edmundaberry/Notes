\section{Fair Gambler's Ruin}
\label{sec:fair}

Let's examine the fair ($p = 0.5$) Gambler's Ruin
problem described in Section \ref{sec:intro}.
Recall that:
\begin{itemize}
  \item The gambler begins with a fortune of $S_0 = i$ dollars.
  \item The gambler plays a series of games, and in each of these games he can either win a dollar with probability $p = 0.5$ or lose a dollar with probability $1-p=0.5$.  
  \item The gambler's winnings in game $t$ are $X_t$.
  \item The gamber's fortune after $n$ games is $S_n = S_0 + \sum\limits_{t = 1}^{n}X_{t}$.
  \item The gambler will stop when he runs out of money ($S_N = 0$) or obtains $N$ dollars ($S_N = N$).
\end{itemize}

First, let's show that $\left\{S_n\right\}$ is a martingale:
\begin{align}
  \mathbb{E}[S_n] &= S_{n-1} + p + (1-p)\cdot(-1) \\
  \mathbb{E}[S_n] &= S_{n-1} + 2p - 1 \\
  \mathbb{E}[S_n] &= S_{n-1} 
\end{align}
So $\left\{S_n\right\}$ is a martingale.

We would like to evaluate the probability that the gambler accumulates $N$ dollars 
and does not go bankrupt ($p'$).  We can evaluate that probability as follows:
\begin{align}
  \mathbb{E}[S_n] &= p' \cdot N + (1-p') \cdot ( 0 ) \\
  \mathbb{E}[S_n] &= p' \cdot N 
\end{align}
The fact that $\left\{S_n\right\}$ is a martingale allows
us to evaluate this probability using the optional stopping theorem:
\begin{align}
  \mathbb{E}[S_n] &= \mathbb{E}[S_0] \\
  \mathbb{E}[S_n] &= i 
\end{align}
Combining this information, we find:
\begin{align}
  \mathbb{E}[S_n] &= p' \cdot N = i \\
  p' &= \frac{i}{N} 
\end{align}
This makes some intuitive sense.  The larger the gambler's initial 
fortune ($i$), the less likely he is to go bankrupt.
Similar, the larger his goal ($N$), the less likely he is to achieve it.

We would also like to find the expected number of games ($\tau$) it will take the 
gambler to either go bankrupt ($S_\tau = 0$) or reach his goal ($S_\tau = N$).
To do this, we must prove that $\left\{S_n^2 - n\right\}$ is also a martingale:
\begin{align}
  \mathbb{E}[S_{n+1}^2 - (n+1)] &= \frac{1}{2}\left(S_n + 1\right)^2 + \frac{1}{2}\left(S_n - 1\right)^2 - (n+1) \\
  \mathbb{E}[S_{n+1}^2 - (n+1)] &= \frac{1}{2}\left(2S_n^2 + 2\right) - (n+1) \\
  \mathbb{E}[S_{n+1}^2 - (n+1)] &= S_n^2 - n 
\end{align}
So $\left\{S_n^2 - n\right\}$ is also a martingale.

Since $\left\{S_n^2 - n\right\}$ is a martingale, we will treat $\tau$ as a stopping time.
Let's evaluate the expectation value of $S_n^2 - n$ at the stopping time, $\tau$:
\begin{align}
  \mathbb{E}[S_\tau^2 - \tau] &= \mathbb{E}[S_\tau^2] - \mathbb{E}[\tau] \\
  \mathbb{E}[S_\tau^2 - \tau] &= \mathbb{E}[p'\cdot \left(N^2\right) + (1-p')\cdot\left(0\right)] - \mathbb{E}[\tau] \\ 
  \mathbb{E}[S_\tau^2 - \tau] &= p'\cdot \left(N^2\right) - \mathbb{E}[\tau]
\end{align}
We have already solved for $p'$, and we plug in the value we derived:
\begin{align}
  \mathbb{E}[S_\tau^2 - \tau] &= \left(\frac{i}{N}\right)\cdot \left(N^2\right) - \mathbb{E}[\tau] \\
  \mathbb{E}[S_\tau^2 - \tau] &= iN - \mathbb{E}[\tau]
\end{align}
Furthermore, since $\left\{S_n^2 - n\right\}$ is a martingale, we can also make use of the optional stopping theorem:
\begin{align}
  \mathbb{E}[S_\tau^2 - \tau] &= \mathbb{E}[S_0^2 - 0] \\
  \mathbb{E}[S_\tau^2 - \tau] &= i^2
\end{align}
Combining this information, we find:
\begin{align}
  \mathbb{E}[S_\tau^2 - \tau] & = iN - \mathbb{E}[\tau] = i^2 \\ 
  \mathbb{E}[\tau] &= i\cdot(N - i) 
\end{align}
So the expected number of games the gambler would have to play to either win or lose
is equal to the product of the number of dollars he would have to lose to go bankrupt ($i$)
and the number of dollars he would have to win to reach his goal ($N-i$).
